\documentclass[12pt]{article}

\usepackage[T1,T2A]{fontenc}
\usepackage[utf8]{inputenc}
\usepackage[bulgarian]{babel}
\usepackage{graphicx}
\usepackage{amssymb}
\usepackage{amsmath}
\usepackage{commath}
\usepackage{float}
\usepackage{booktabs}   
\usepackage{ltablex}

\parindent 0px % turn off indenting

% have nicer-looking links
\usepackage{hyperref}
\hypersetup{
  colorlinks   = true, %Colours links instead of ugly boxes
  urlcolor     = red, %Colour for external hyperlinks
  linkcolor    = blue, %Colour of internal links
  citecolor   = blue
}
\usepackage[hypcap=true]{caption}


%%%%%%%%%%%%%%%%%%%%%%%%%%%%%%%%%%%%%%%%%%%%%%%%%%
%%%%%%%%%%%%%%%% START OF PREAMBLE %%%%%%%%%%%%%%%
%%%%%%%%%%%%%%%%%%%%%%%%%%%%%%%%%%%%%%%%%%%%%%%%%%

\newcommand{\JMUTitle}[9]{

  \thispagestyle{empty}
  \vspace*{\stretch{1}}
  {\parindent0cm
  \rule{\linewidth}{.7ex}}
  \begin{flushright}
    \vspace*{\stretch{1}}
    \sffamily\bfseries\Huge
    #1\\
    \vspace*{\stretch{1}}
    \sffamily\bfseries\large
    #2\\
    \vspace*{\stretch{1}}
    \sffamily\bfseries\small
    #3
  \end{flushright}
  \rule{\linewidth}{.7ex}

  \vspace*{\stretch{1}}
  \begin{center}
    \includegraphics[width=3in]{./images/logo.png} \\
    \vspace*{\stretch{1}}
    \Large Курсов проект по \\ \textit{Подходи за обработка на естествен език} \\

    \vspace*{\stretch{2}}
    \large Факултет по математика и информатика\\
    \large Софийски университет\\
    
    \vspace*{\stretch{1}}
    \large Лектор: #8 \\[1mm]
    
    \vspace*{\stretch{1}}
    \large #7 \\

  \end{center}
}


%%%%%%%%%%%%%%%%%%%%%%%%%%%%%%%%%%%%%%%%%%%%%%%%%%
%%%%%%%%%%%%%%%%% END OF PREAMBLE %%%%%%%%%%%%%%%%
%%%%%%%%%%%%%%%%%%%%%%%%%%%%%%%%%%%%%%%%%%%%%%%%%%


\begin{document}  

  \JMUTitle
      {Ati: Система за разпознаване на авторство}
      {Симеон Христов}
      {6MI3400191}
      
      {Wirtschaftswissenschaftlichen Fakultät}  % Name der Fakultaet
      {W"urzburg 2018}                          % Ort und Jahr der Erstellung
      {Февруари 2023}                              % Tag der Abgabe
      {проф. дмн Галя Ангелова}               % Name des Erstgutachters
      {Zweitgutachter}                          % Name des Zweitgutachters

  \clearpage

\clearpage


%%%%%%%%%%%%%%%%%%%%%%%%%%%%%%%%%%%%%%%%%%%%%%%%%%
% Въведение
%%%%%%%%%%%%%%%%%%%%%%%%%%%%%%%%%%%%%%%%%%%%%%%%%%

\section{Въведение}

Целта на проекта е създаването на система за категоризация, която при даден корпус от документи - $D$, всеки от които е написан от един автор $y$, идентифицира автора на анонимен текст $x$.
    

%%%%%%%%%%%%%%%%%%%%%%%%%%%%%%%%%%%%%%%%%%%%%%%%%%
% Данни
%%%%%%%%%%%%%%%%%%%%%%%%%%%%%%%%%%%%%%%%%%%%%%%%%%

\section{Данни}



%%%%%%%%%%%%%%%%%%%%%%%%%%%%%%%%%%%%%%%%%%%%%%%%%%
% Метод
%%%%%%%%%%%%%%%%%%%%%%%%%%%%%%%%%%%%%%%%%%%%%%%%%%

\section{Метод}


Изискванията за разработване на системата са:

\begin{enumerate}
    \item Създаване на паяк, копаещ документи (текстовете на съответните автори) от уеб страница.
    \item Създаване на модел, който е трениран върху тренировъчно множество ($80\%$ от данните), валидиран върху валидационно множество ($10\%$ от оставащи данни) и оценен върху тестово множество (последните $10\%$ оставащи данни).
    \item Сравняване на поне 3 стилистични метрики за всички автори.
    \item Сравняване на различни представяния на текст: \textit{tf-tdf} и \textit{transformer sentence embeddings}.
    \item Сравняване на класификатори: един и ансамбъл.
    \item Създаване на потребителски интерфейс.
\end{enumerate}

    
%%%%%%%%%%%%%%%%%%%%%%%%%%%%%%%%%%%%%%%%%%%%%%%%%%
% Експерименти
%%%%%%%%%%%%%%%%%%%%%%%%%%%%%%%%%%%%%%%%%%%%%%%%%%

\section{Експерименти}

\subsection{Използвани технологии, платформи и библиотеки}

[ ] TODO: Подходящи средства за реализация за проекта (технологии, платформи и библиотеки). Избор на средствата и начин за използването им;

% describe the three levels for the tokenize functions.

\subsection{Реализация/Провеждане на експерименти}

[ ] TODO: Реализация (на модулите); 
За система/приложение: На кратко: планиране на тестването - тестови сценарии,...; Анализ на резултатите.


%%%%%%%%%%%%%%%%%%%%%%%%%%%%%%%%%%%%%%%%%%%%%%%%%%
% Резултати. Дискусия
%%%%%%%%%%%%%%%%%%%%%%%%%%%%%%%%%%%%%%%%%%%%%%%%%%

\section{Резултати. Дискусия}


%%%%%%%%%%%%%%%%%%%%%%%%%%%%%%%%%%%%%%%%%%%%%%%%%%
% Заключение и бъдеща работа
%%%%%%%%%%%%%%%%%%%%%%%%%%%%%%%%%%%%%%%%%%%%%%%%%%

\section{Заключение и бъдеща работа}

Разработена е система, която може да класифицира даден текст или откъс от текст на базата на различни негови представяния.

Разработената система може да бъде основата за разработване на приложение, което да:

\begin{enumerate}
    \item \textbf{Проверява на авторство}: Дали даден текст наистина е написан от определен автор?
    \item \textbf{Открива плагиатство}: Намиране на прилики между два или повече текста;
    \item \textbf{Създава профил или характеризира на даден автора}: Извличане на информация за възрастта, образованието, пола и т.н. на автора на даден текст;
    \item \textbf{Открива стилистични несъответствия} (както може да се случи при съвместно писане): Дали авторът настина е само един?
    \item \textbf{Отговоря на въпроси}: В матурата по български език и литература има въпроси, които са фокусирани върху разпознаване на автора на даден отказ или разпознаване на автора, който пише за определен герой.
\end{enumerate}

%%%%%%%%%%%%%%%%% END OF MAIN TEXT %%%%%%%%%%%%%%%%

\end{document}